%---------- Inleiding ---------------------------------------------------------

\section{Introductie} % The \section*{} command stops section numbering
\label{sec:introductie}
Binnen TVH is er dus heel wat input van data in de vorm van microservices. Heel wat componenten hebben deze data nodig. Maar het spreekt voor zich dat niet ieder component, alle data nodig heeft. Daarom maakt dit bedrijf voornamelijk gebruik van de technologie \emph{Kafka} om deze microservices te verwerken en door te geven naar de juiste componenten. Dit onderzoek zal nagaan of \emph{Kafka} inderdaad wel de beste technologie is om al deze data te verwerken. Aan de hand van deze vragen komt dit onderzoek hopelijk tot een besluit welke technologie het meest geschikt is:
\begin{itemize}
\item Welke technologie kan het snelst kleine hoeveelheid data verwerken?
\item Welke technologie kan het snelst grotere hoeveelheid data verwerken?
\item Welke technologie is het meest gebruiksvriendelijk?
\end{itemize}


%---------- Stand van zaken ---------------------------------------------------

\section{Literatuurstudie}
\label{sec:Literatuurstudie}
In het onderzoek van \textcite{Shadija2017} staat te lezen dat microservices de business analysts helpen om grote schaalbare applicaties te maken. Het grote voordeel hiervan is flexibiliteit. Als er nieuwe functionaliteiten moeten worden gemaakt dan is het door de microservices gemakkelijk te implementeren. Vooral in het Internet of Things (IoT) domein kan dit het werk versoepelen. Dus voor het bedrijf TVH lijkt het de meest geschikte manier om de input van al de verzamelde data te verwerken.

Ook in andere onderzoeken naar microservices wordt \emph{Kafka} gebruikt. Zoals in het onderzoek van \textcite{Khazaei2017}. Als we kijken naar de conclusie uit dit onderzoek, blijkt dat \emph{Kafka} in grote lijnen het best scoort. De andere technologieën die in dit onderzoek gebruikt werden zijn \emph{Spark} en \emph{Cassandra}.

Het verschil van dit onderzoek met het onderzoek van \textcite{Shadija2017} en het onderzoek van \textcite{Khazaei2017} is dat dit onderzoek nagaat welke technologie het beste is wanneer er veel data is voor het bedrijf TVH specifiek. Het onderzoek van \textcite{Khazaei2017} sluit hier het dichtst bij aan maar is wat te uitgebreid.

Ook \textcite{Nycander2015} en \textcite{Cherradi2017} behandelen microservices in hun onderzoek.


% Voor literatuurverwijzingen zijn er twee belangrijke commando's:
% \autocite{KEY} => (Auteur, jaartal) Gebruik dit als de naam van de auteur
%   geen onderdeel is van de zin.
% \textcite{KEY} => Auteur (jaartal)  Gebruik dit als de auteursnaam wel een
%   functie heeft in de zin (bv. ``Uit onderzoek door Doll & Hill (1954) bleek
%   ...'')


%---------- Methodologie ------------------------------------------------------
\section{Methodologie}
\label{sec:methodologie}
Om te bepalen welke technologie het beste is bij het gebruiken van microservices, zal dit onderzoek de verschillende technologieën vergelijken. Er zal gekeken welke technologie het snelste data kan opvragen en verwerken. Ook zal het onderzoek mensen die met verschillende technologieën reeds gewerkt hebben vragen naar hun voorkeur. 

%---------- Verwachte resultaten ----------------------------------------------
\section{Verwachte resultaten}
\label{sec:verwachte_resultaten}
Het resultaat van dit onderzoek zal hopelijk aanwijzen welke technologie het meest geschikt is voor deze hoeveelheid van data aan te kunnen. We hopen dit te zien door cijfergegevens van de snelheden van uitvoering. 


%---------- Verwachte conclusies ----------------------------------------------
\section{Verwachte conclusies}
\label{sec:verwachte_conclusies}

Aangezien TVH al gebruik maakt van \emph{Kafka}, en ook in andere onderzoeken \emph{Kafka} gebruikt werd of bestempeld werd als beste oplossing, kunnen we in dit onderzoek hopelijk ook concluderen dat \emph{Kafka} de beste oplossing is om met microservices om te gaan.

