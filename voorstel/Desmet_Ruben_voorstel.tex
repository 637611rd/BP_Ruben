%==============================================================================
% Sjabloon onderzoeksvoorstel bachelorproef
%==============================================================================
% Gebaseerd op LaTeX-sjabloon ‘Stylish Article’ (zie voorstel.cls)
% Auteur: Jens Buysse, Bert Van Vreckem

\documentclass[fleqn,10pt]{voorstel}

%------------------------------------------------------------------------------
% Metadata over het voorstel
%------------------------------------------------------------------------------

\JournalInfo{HoGent Bedrijf en Organisatie}
\Archive{Bachelorproef 2018 - 2019} % Of: Onderzoekstechnieken

%---------- Titel & auteur ----------------------------------------------------

% TODO: geef werktitel van je eigen voorstel op
\PaperTitle{Welke technologie is het meest geschikt om in een IoT-applicatie data van microservices te gebruiken?}
\PaperType{Onderzoeksvoorstel Bachelorproef} % Type document

% TODO: vul je eigen naam in als auteur, geef ook je emailadres mee!
\Authors{Ruben Desmet\textsuperscript{1}} % Authors
\CoPromotor{Nog niet bepaald\textsuperscript{2} (TVH)}
\affiliation{\textbf{Contact:}
    \textsuperscript{1} \href{mailto:ruben.desmet.y7611@student.hogent.be}{ruben.desmet.y7611@student.hogent.be};
    \textsuperscript{2} \href{
    }{};
}

%---------- Abstract ----------------------------------------------------------

\Abstract{Het bedrijf TVH heeft veel hoogwerkers en andere soort werktuigen in bezit. Van deze werktuigen komt er veel data binnen. Ze maken gebruik van microservices om deze data te verwerken en door te geven naar de componenten die deze data nodig hebben. In deze Internet of Things-applicatie vraagt de component alleen de input op die hij nodig heeft. De technologie die hierbij gebruikt werd om deze data door te spelen was \emph{RabbitMq}. Nu zijn ze aan het overschakelen naar een andere technologie genaamd \emph{Kafka}. Deze bachelorproef zal uitzoeken welke technologieën mogelijke alternatieven zijn voor \emph{Kafka} en \emph{RabbitMq}, wat de ervaringen zijn van gebruikers en de technologieën zullen met elkaar vergeleken worden. Om te vergelijken maakt dit onderzoek gebruik van de voor- en nadelen rekening houdende met de ervaring van de ondervraagde gebruikers. Ook zal er op een virtuele machine gekeken worden welke technologie het best presteert op basis van snelheid. Deze stap zal enkele keren met verschillende data uitgevoerd worden. De bedoeling van dit onderzoek is dat het uitwijst welke technologie het meest geschikt is om te werken met microservices in het bedrijf TVH. De data die hiervoor zal gebruikt worden, is data verkregen van verschillende werktuigen.
    
    
}

%---------- Onderzoeksdomein en sleutelwoorden --------------------------------
% TODO: Sleutelwoorden:
%
% Het eerste sleutelwoord beschrijft het onderzoeksdomein. Je kan kiezen uit
% deze lijst:
%
% - Mobiele applicatieontwikkeling
% - Webapplicatieontwikkeling
% - Applicatieontwikkeling (andere)
% - Systeembeheer
% - Netwerkbeheer
% - Mainframe
% - E-business
% - Databanken en big data
% - Machineleertechnieken en kunstmatige intelligentie
% - Andere (specifieer)
%
% De andere sleutelwoorden zijn vrij te kiezen

\Keywords{Microservices --- IoT-Applicatie --- Kafka --- RabbitMq} % Keywords
\newcommand{\keywordname}{Sleutelwoorden} % Defines the keywords heading name

%---------- Titel, inhoud -----------------------------------------------------

\begin{document}
    
    \flushbottom % Makes all text pages the same height
    \maketitle % Print the title and abstract box
    \tableofcontents % Print the contents section
    \thispagestyle{empty} % Removes page numbering from the first page
    
    %------------------------------------------------------------------------------
    % Hoofdtekst
    %------------------------------------------------------------------------------
    
    % De hoofdtekst van het voorstel zit in een apart bestand, zodat het makkelijk
    % kan opgenomen worden in de bijlagen van de bachelorproef zelf.
    %---------- Inleiding ---------------------------------------------------------

\section{Introductie} % The \section*{} command stops section numbering
\label{sec:introductie}
Binnen TVH is er dus heel wat input van data die via microservices naar de juiste componenten verstuurd worden. Het spreekt voor zich dat niet ieder component, alle data nodig heeft. Daarom maakt dit bedrijf voornamelijk gebruik van de technologie \emph{Kafka} om met deze microservices te werken. Dit onderzoek zal nagaan of \emph{Kafka} inderdaad wel de beste technologie is om al deze data te verwerken in dit bedrijf. Aan de hand van deze onderzoeksvraag en deelvragen komt dit onderzoek hopelijk tot een besluit welke technologie het meest geschikt is:
\begin{itemize}
    \item Welke technologie is het best om met microservices te werken voor het bedrijf TVH?
    \begin{itemize}
        \item Bestaan er nog alternatieven voor \emph{Kafka} en \emph{RabbitMq}?
        \item Wat zijn de bevindingen van gebruikers?
        \item Welke technologie is het snelst?
    \end{itemize}
\end{itemize}


%---------- Stand van zaken ---------------------------------------------------

\section{Literatuurstudie}
\label{sec:Literatuurstudie}
In het onderzoek van \textcite{Shadija2017} staat te lezen dat microservices de business analysts helpen om grote schaalbare applicaties te maken. Het grote voordeel hiervan is flexibiliteit. Als er nieuwe functionaliteiten moeten worden gemaakt dan is het door de microservices gemakkelijk te implementeren. Vooral in het Internet of Things (IoT) domein kan dit het werk versoepelen. Dus voor het bedrijf TVH lijkt het de meest geschikte manier om de input van al de verzamelde data te verwerken.

Ook in andere onderzoeken naar microservices wordt \emph{Kafka} gebruikt. Zoals in het onderzoek van \textcite{Khazaei2017}. Als we kijken naar de conclusie uit dit onderzoek, blijkt dat \emph{Kafka} in grote lijnen het best scoort. De andere technologieën die in dit onderzoek gebruikt werden zijn \emph{Spark} en \emph{Cassandra}.

Het verschil van deze bachelorproef-onderzoek met het onderzoek van \textcite{Shadija2017} en het dat van \textcite{Khazaei2017} is dat dit onderzoek nagaat welke technologie het beste is voor het bedrijf TVH. Het besluit van dit onderzoek is dus niet noodzakelijk een algemeen besluit voor alle bedrijven die met microservices werken. Het onderzoek van \textcite{Khazaei2017} sluit hier het dichtst bij aan omdat het ook \emph{Kafka} en andere technologiën vergelijkt, maar het onderzoek legt meer de nadruk hoe flexibel een programmeerbaar, zelf-besturend IoT-platform is, gebruik makend van microservices. Het vergelijken van verschillende technologieën bij \textcite{Khazaei2017} is dus maar een klein onderdeel van het onderzoek en wordt bovendien in een andere architectuur toegepast zoals de titel meedeelt: `\emph{SAVI-IoT: A Self-Managing Containerized IoT Platform}`. De meerwaarde van deze vergelijking voor de conclusie is dus niet zo groot voor \textcite{Khazaei2017}.

Ook \textcite{Nycander2015} en \textcite{Cherradi2017} behandelen microservices in hun onderzoek.


% Voor literatuurverwijzingen zijn er twee belangrijke commando's:
% \autocite{KEY} => (Auteur, jaartal) Gebruik dit als de naam van de auteur
%   geen onderdeel is van de zin.
% \textcite{KEY} => Auteur (jaartal)  Gebruik dit als de auteursnaam wel een
%   functie heeft in de zin (bv. ``Uit onderzoek door Doll & Hill (1954) bleek
%   ...'')


%---------- Methodologie ------------------------------------------------------
\section{Methodologie}
\label{sec:methodologie}
Om te bepalen welke technologie het beste is bij het gebruiken van microservices, zal dit onderzoek de verschillende technologieën vergelijken. Eerst zal er een rondvraag gehouden worden over de bevindingen en de voor- en nadelen van \emph{Kafka} en \emph{RabbitMq}. Er wordt ook gepolst of medewerkers met nog andere technologieën reeds gewerkt hebben en wat daar de bevindingen zijn. Er zal voornamelijk gewerkt worden met open vragen waardoor er veel nieuwe nuttige informatie zal ontstaan voor dit onderzoek.

Dan zal er onderzocht worden of er effectief nog alternatieven bestaan voor \emph{Kafka} en \emph{RabbitMq}.

Als laatste zal op een virtuele machine de realiteit nagebootst worden. Dit wordt gedaan door elke technologie op deze virtuele machine te zetten en daarna te gaan meten wat de snelheden zijn bij het opvragen, verwerken, ... van voorbeelddata.

%---------- Verwachte resultaten ----------------------------------------------
\section{Verwachte resultaten}
\label{sec:verwachte_resultaten}
Het resultaat van dit onderzoek zal hopelijk aanwijzen welke technologie het meest geschikt is om deze hoeveelheid van data aan te kunnen. We hopen dit te zien door cijfergegevens van de snelheden van uitvoering. 


%---------- Verwachte conclusies ----------------------------------------------
\section{Verwachte conclusies}
\label{sec:verwachte_conclusies}

Aangezien TVH al gebruik maakt van \emph{Kafka}, en ook in andere onderzoeken \emph{Kafka} gebruikt werd of bestempeld werd als beste oplossing, kunnen we in dit onderzoek hopelijk ook concluderen dat \emph{Kafka} de beste oplossing is om met microservices om te gaan binnen TVH met hun specifieke data.


    
    %------------------------------------------------------------------------------
    % Referentielijst
    %------------------------------------------------------------------------------
    % TODO: de gerefereerde werken moeten in BibTeX-bestand ``voorstel.bib''
    % voorkomen. Gebruik JabRef om je bibliografie bij te houden en vergeet niet
    % om compatibiliteit met Biber/BibLaTeX aan te zetten (File > Switch to
    % BibLaTeX mode)
    
    \phantomsection
    \printbibliography[heading=bibintoc]
    
\end{document}
