%%=============================================================================
%% Voorwoord
%%=============================================================================

\chapter*{Woord vooraf}
\label{ch:voorwoord}

%% TODO:
%% Het voorwoord is het enige deel van de bachelorproef waar je vanuit je
%% eigen standpunt (``ik-vorm'') mag schrijven. Je kan hier bv. motiveren
%% waarom jij het onderwerp wil bespreken.
%% Vergeet ook niet te bedanken wie je geholpen/gesteund/... heeft

Tijdens dat ik dit onderzoek voerde, liep ik ook mijn stage in het bedrijf TVH. Al snel dit academiejaar kwam ik te weten dat ik hier mijn eerste werkervaring mocht opdoen. Toen er gevraagd werd voor een onderwerp te zoeken voor deze bachelorproef, leek het mij handig om een onderzoek te voeren die interessant was voor TVH. Na een brainstorm-sessie met mijn stagementor en de product-owner van mijn stage kwam ik tot dit onderwerp. 

Eerst wist ik nog helemaal niets over dit onderwerp. Een ideaal moment dus om zaken bij te leren. Op mijn stage werken we ook met een event-bus. Ons team van de stage gebruikt die van Google, namelijk Google Pub/Sub. Andere teams gebruiken Kafka en ik vernam dat vroeger ook nog RabbitMq gebruikt werd. Dit leek mij ideaal om te onderzoeken welke technologie nu eigenlijk beter is.

Gelukkig waren er een tal van mensen die mij steunden tijdens deze periode. Eerst en vooral wil ik mijn ouders bedanken, die altijd klaar stonden indien ze met iets konden helpen. Verder wil ik ook mijn collega's van mijn stage bedanken voor ideeën aan te reiken. Verder zijn er nog twee personen die voor dit onderzoek een dikke pluim verdienen. Als eerste wil ik mevrouw Vandermeersch heel erg bedanken om mij steeds vooruithelpende feedback te geven en altijd bereid te zijn om alles eens na te lezen. Als laatste wil ik ook een hele grote dankjewel zeggen aan Maarten Meersseman die mij inhoudelijk bijstond voor de technologieën van Kafka en RabbitMq, en altijd klaar stond om extra uitleg te geven. Door deze personen heb ik ook meer inzicht gekregen in het onderzochte thema.