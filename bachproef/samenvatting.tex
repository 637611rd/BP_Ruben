%%=============================================================================
%% Samenvatting
%%=============================================================================

% TODO: De "abstract" of samenvatting is een kernachtige (~ 1 blz. voor een
% thesis) synthese van het document.
%
% Deze aspecten moeten zeker aan bod komen:
% - Context: waarom is dit werk belangrijk?
% - Nood: waarom moest dit onderzocht worden?
% - Taak: wat heb je precies gedaan?
% - Object: wat staat in dit document geschreven?
% - Resultaat: wat was het resultaat?
% - Conclusie: wat is/zijn de belangrijkste conclusie(s)?
% - Perspectief: blijven er nog vragen open die in de toekomst nog kunnen
%    onderzocht worden? Wat is een mogelijk vervolg voor jouw onderzoek?
%
% LET OP! Een samenvatting is GEEN voorwoord!

%%---------- Nederlandse samenvatting -----------------------------------------
%
% TODO: Als je je bachelorproef in het Engels schrijft, moet je  een
% Nederlandse samenvatting invoegen. Haal daarvoor onderstaande code uit
% commentaar.
% Wie zijn bachelorproef in het Nederlands schrijft, kan dit negeren, de inhoud
% wordt niet in het document ingevoegd.

\IfLanguageName{english}{%
\selectlanguage{dutch}
\chapter*{Samenvatting}
\lipsum[1-4]
\selectlanguage{english}
}{}

%%---------- Samenvatting -----------------------------------------------------
% De samenvatting in de hoofdtaal van het document

\chapter*{\IfLanguageName{dutch}{Samenvatting}{Abstract}}
Dit onderzoek vergelijkt drie verschillende technologieën die data verzenden en ontvangen via event-bussen. Bij TVH zijn ze al reeds in contact gekomen met de drie technologieën die hier onderzocht worden. Het is belangrijk dat dit onderzocht werd omdat men dan eens kon te weten komen als de meeste teams effectief met de beste technologie bezig zijn. Momenteel gebruiken de meeste teams Kafka, maar bijvoorbeeld in het team waar de onderzoeker (Ruben Desmet) zijn stage gelopen heeft, werd er Google Pub/Sub gebruikt. Het komt dus niet ongelegen om te weten te komen welke nu eigenlijk het beste gebruikt moet worden.

Dit zou kunnen tot positieve gevolgen leiden. Indien ze een meer efficiëntere technologie vinden, kunnen ze sneller of met minder resources hun data verzenden. Hierdoor kunnen er mogelijk andere voordelen naar boven komen, zoals bijvoorbeeld meer data verwerken binnen een tijdspanne. Indien de conclusie van dit onderzoek zou zijn dat ze al reeds met de beste technologie bezig zijn, dan is dit ook een positieve uitkomst. Hierdoor weten de teams dat ze nog geen tijd verloren zijn.

Voor dit onderzoek zijn Google Pub/Sub, Kafka en RabbitMq met elkaar vergeleken. Eerst werd de snelheid gemeten tussen het verzenden naar de event-bus en het ontvangen ervan in een andere applicatie. Dit werd voor drie verschillende hoeveelheden van objecten gedaan. Als eerste werden er 10 000 objecten genomen, daarna 100 000 en als laatste 1 000 000. Hierna werd ook nog het memory gebruik vergeleken. Dit werd maar voor één hoeveelheid gedaan, namelijk 1 000 objecten, omdat dit veel meer tijd in beslag nam. De resultaten van de memory en de snelheid werden niet op het zelfde moment berekend. 

Dit document beschrijft hoe dit onderzoek in zijn werk is gegaan. De resultaten uit dit onderzoek waren bij grotere hoeveelheden moeilijk te vergelijken met elkaar omdat er verschillende hoeveelheden objecten effectief aankwamen. Maar uit het onderzoek kunnen we toch concluderen dat Kafka het beste is om mee te werken voor TVH. Als verder onderzoek zou je dit onderzoek meerdere keren kunnen uitvoeren. Een andere mogelijkheid is om de applicaties niet lokaal maar in de cloud te laten draaien, op die manier ben je niet afhankelijk van een pc of laptop.
