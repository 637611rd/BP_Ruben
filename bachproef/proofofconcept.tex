%%=============================================================================
%% Methodologie
%%=============================================================================

\chapter{Proof of Concept}
\label{ch:proof-of-concept}
\lstset{
    tabsize = 4, %% Sets tab space width.
    showstringspaces = false, %% Prevents space marking in strings, string is defined as the text that is generally printed directly to the console.
    numbers = left, %% Displays line numbers on the left.
    commentstyle = \color{green}, %% Sets comment color.
    keywordstyle = \color{blue}, %% Sets  keyword color.
    stringstyle = \color{red}, %% Sets  string color.
    rulecolor = \color{black}, %% Sets frame color to avoid being affected by text color.
    basicstyle = \small \ttfamily , %% Sets listing font and size.
    breaklines = true, %% Enables line breaking.
    numberstyle = \tiny,
}

%%\begin{lstlisting}[language = Java , frame = trBL , firstnumber = last , escapeinside={(*@}{@*)}]
%%public class Factorial
%%{
%%public static void main(String[] args)
%%{   final int NUM_FACTS = 100;
%%for(int i = 0; i < NUM_FACTS; i++)
%%System.out.println( i + "! is " + factorial(i));
%%}
%%
%%public static int factorial(int n)
%%{   int result = 1;
%%for(int i = 2; i <= n; i++) (*@\label{for}@*)
%%result *= i;
%%return result;
%%}
%%}
%%\end{lstlisting}


%% TODO: Hoe ben je te werk gegaan? Verdeel je onderzoek in grote fasen, en
%% licht in elke fase toe welke stappen je gevolgd hebt. Verantwoord waarom je
%% op deze manier te werk gegaan bent. Je moet kunnen aantonen dat je de best
%% mogelijke manier toegepast hebt om een antwoord te vinden op de
%% onderzoeksvraag.

% TODO: uitleg wat we allemaal nodig hebben, producer uitleggen, consumer uitleggen, 3 verschillende groottes uitleggen.
